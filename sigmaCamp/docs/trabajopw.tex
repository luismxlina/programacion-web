\documentclass{article}
\usepackage[utf8]{inputenc}
\usepackage{graphicx}
\usepackage{tocloft}

\renewcommand{\cftsecleader}{\cftdotfill{\cftdotsep}}

\begin{document}
	
	\begin{titlepage}
		\centering
		\vspace*{2cm}
		%\includegraphics[width=0.3\textwidth]{tu_imagen.jpg} % Añade la ruta de tu imagen
		\vspace{1cm}
		\Huge\textbf{Práctica 1 Programación Web}
		\vfill
		\vfill
	\end{titlepage}
	
	\tableofcontents
	
	\newpage
	
	\section{Organización}
	Para la realización de la práctica el grupo se ha reunido varias veces. En estas reuniones se diseñó el modelo de la aplicación (diagrama UML) para tener una visual clara y precisa de lo que se pide. Además, se repartieron las tareas de forma equitativa usando Trello para el control de estas. Para trabajar en equipo se ha usado GitHub para subir las versiones del código. En ciertas ocasiones se ha usado Pair Programming en Eclipse y en otras se han usado las ramas de GitHub y, posteriormente, realizando pull requests para aplicar cambios en la rama master.
	
	\section{Implementación de la Práctica}
	\subsection{Estructura de Carpetas}
	Para la estructura de carpetas se ha seguido como modelo el ejemplo subido en Moodle. Podemos diferenciar varias carpetas:
	
	\begin{itemize}
		\item \textbf{business}: se corresponde con los casos de uso y la lógica de negocio de la aplicación. Dentro de ella hay dos carpetas más, \textbf{factories} (se corresponde con el patrón factoría del ejercicio 2) y \textbf{handlers} (se corresponde con los gestores del ejercicio 3).
		\item \textbf{data}: se guardan los modelos y los enums de la aplicación. Los cuales son accedidos a través de los handlers.
	\end{itemize}
	
	\subsection{Patrones de Diseño}
	\subsubsection{Patrón Factoría}
	Se ha seguido el guión de la práctica y se han analizado los ejemplos vistos en clase para implementar el patrón en la aplicación. En este se nos pedía poder crear reservas individuales o bonos (estos se han definido como una clase, y almacenan las IDs de las reservas que se hacen a través de la factoría creadora de los bonos).
	
	\subsubsection{Patrón Singleton}
	Este patrón no estaba especificado en la práctica, pero creemos que es una buena práctica el uso en los manejadores (gestores de usuarios, circuitos y reservas), ya que la información debe mantenerse en una sola instancia (tener más de una podría dar lugar a no mantener la integridad de la información).
	
	\subsection{Otras Decisiones de Importancia}
	\subsubsection{Reservas de Bono}
	Estas reservas se crean, validan y se introducen dentro del manejador de reservas. La diferencia con la reserva individual reside en pertenecer a un bono. Para la implementación de esto, el grupo decidió la creación de una clase Bono. Esta clase define un array de 5 elementos, guardando las IDs de las reservas, permitiendo así no tener que modificar la instancia de la clase bono si en algún momento se modificase la reserva, ya que el ID de la reserva no cambia.
	
	Por otra parte, se han usado ficheros binarios, ya que permiten el volcado de información y recarga del fichero a la clase de forma muy rápida, y sin entrar en la creación de funciones de lectura ni escritura muy complejas. Destacar también el uso del fichero data.properties para la definición del nombre de los ficheros, y el uso de la función loadFilesPath para definir las rutas de los ficheros usados.
	
	\section{Dificultades Encontradas}
	A continuación vamos a comentar acerca de las siguientes dificultades o problemas encontrados a la hora de diseñar e implementar la app.
	
	\begin{itemize}
		\item \textbf{Diseño del diagrama UML}: Este diagrama el equipo tardó bastantes días en diseñarlo, ya que hay muchas decisiones que son abiertas. Y muchas decisiones tomadas al principio han sido modificadas a medida que avanzaba el proyecto, por lo que es un continuo cambio en propiedades y métodos de entidades del proyecto. Y obviamente, esto último puede llegar a confundir en determinados momentos.
		\item \textbf{Patrón factoría}: Nos costó diferenciar cuál era el producto y factoría, porque el ejemplo de Moodle nos confundió debido a que posee un ArrayList y nosotros interpretamos la modalidad bono como un ArrayList de modalidades individuales. Gracias a la tutoría con el profesor de prácticas nos dimos cuenta de que lo implementamos bien, pero no sabíamos cómo implementar la modalidad bono.
		\item \textbf{Compilación y ejecución del fichero .jar}: Primero hemos tenido problemas con el JRE usado, ya que un integrante del grupo usa JRE 11 y otro JRE 17, por lo que se lanzaba excepciones continuamente. Y otro problema fue no exportar los paquetes y clases al crear el .jar.
		\item \textbf{Excepciones de ficheros}: Teníamos un problema y es que debido a nuestra implementación, se cargaban primero todos los ficheros de todas las entidades. Pero si el cliente no poseía dichos ficheros o estaban vacíos, se producían 2 tipos de excepciones. Nos costó darnos cuenta de dicha excepción para poder capturarla posteriormente, entender qué había ocurrido y, por tanto, controlar de forma correcta dicho flujo alternativo del programa.
	\end{itemize}
	
	\section{Bibliografía}
	\begin{itemize}
		\item \textbf{Curso de Java}
		\item \textbf{Documentación ArrayList}
		\item \textbf{Documentación Properties}
		\item \textbf{Guía de Ficheros Binarios}
	\end{itemize}
	
\end{document}

